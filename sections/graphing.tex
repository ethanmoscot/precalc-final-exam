\documentclass[../main.tex]{subfiles}
\begin{document}
\section*{Graphing}
    \begin{questions}
    \setcounter{question}{71}
   
    % Units 1 & 2
    \question[1] Given an angle in standard position with a terminal ray that passes through the point $(4,8)$, find the values of the six trigonometric functions. Graph this and label the point and the angle.

        \begin{left} 
        \begin{tikzpicture}[scale=1.3]
        \draw[gray] (-1.5,0) -- (1.5,0);
        \draw[gray] (0,1.5) -- (0,-1.5);
        \end{tikzpicture} 
        \end{left} \vspace{\stretch{0.5}}
        
    \question[1] Given an angle in standard position with a terminal ray that passes through the point $(-2,-5)$, find the values of the six trigonometric functions. Graph this and label the point and the angle.

        \begin{left} 
        \begin{tikzpicture}[scale=1.3]
        \draw[gray] (-1.5,0) -- (1.5,0);
        \draw[gray] (0,1.5) -- (0,-1.5);
        \end{tikzpicture} 
        \end{left} \vspace{\stretch{0.5}}
    
    % Unit 3
    \question[1] Sketch the graph of $f(x) = (x-3)^2 + 4$ and state its domain, range and parent function.
    
        \begin{left} 
        \begin{tikzpicture}[scale=1.3]
        \draw[gray] (-1.5,0) -- (1.5,0);
        \draw[gray] (0,1.5) -- (0,-1.5);
        \end{tikzpicture} 
        \end{left}
        
    Domain: \line(1,0){40} \\
    \newline
    Range: \line(1,0){40} \\
    \newline
    Parent Function: \line(1,0){40} \\
    \vspace{\stretch{0.5}}
    
    \newpage
    \question[1] Sketch the graph of $f(x) = -3|x+1| - 2$ and state its domain, range and parent function.
    
        \begin{left} 
        \begin{tikzpicture}[scale=1.3]
        \draw[gray] (-1.5,0) -- (1.5,0);
        \draw[gray] (0,1.5) -- (0,-1.5);
        \end{tikzpicture} 
        \end{left}
        
    Domain: \line(1,0){40} \\
    \newline
    Range: \line(1,0){40} \\
    \newline
    Parent Function: \line(1,0){40} \\
    \vspace{\stretch{0.5}}
    
    \question[1] Sketch the graph of $f(x) = \frac{3}{x - 4}$ and state its domain, range and parent function.
    
        \begin{left} 
        \begin{tikzpicture}[scale=1.3]
        \draw[gray] (-1.5,0) -- (1.5,0);
        \draw[gray] (0,1.5) -- (0,-1.5);
        \end{tikzpicture} 
        \end{left}
        
    Domain: \line(1,0){40} \\
    \newline
    Range: \line(1,0){40} \\
    \newline
    Parent Function: \line(1,0){40} \\
    \vspace{\stretch{0.5}}
    
    \question[1] Sketch the graph of the rational function $\frac{x}{x^2 - 4}$ and find its end behavior and vertical asymptote(s).
    
        \begin{left} 
        \begin{tikzpicture}[scale=1.3]
        \draw[gray] (-1.5,0) -- (1.5,0);
        \draw[gray] (0,1.5) -- (0,-1.5);
        \end{tikzpicture} 
        \end{left}
    
    $\lim_{x\to\infty} f(x)$ =
    \newline    
    $\lim_{x\to-\infty} f(x)$ =
    \newline
    Vertical Asymptote(s): \line(1,0){40} \\
    \vspace{\stretch{0.5}}
    
    \question[1] Sketch the graph of the rational function $\frac{2x - 2}{x - 4}$ and find its end behavior and vertical asymptote(s).
    
        \begin{left} 
        \begin{tikzpicture}[scale=1.3]
        \draw[gray] (-1.5,0) -- (1.5,0);
        \draw[gray] (0,1.5) -- (0,-1.5);
        \end{tikzpicture} 
        \end{left}
    
    $\lim_{x\to\infty} f(x)$ =
    \newline
    $\lim_{x\to-\infty} f(x)$ =
    \newline
    Vertical Asymptote(s): \line(1,0){40} \\
    \vspace{\stretch{0.5}}
    
    \question[1] Sketch the graph of the rational function $\frac{x^3}{x^2 - 16}$ and find its end behavior and vertical asymptote(s).
    
        \begin{left} 
        \begin{tikzpicture}[scale=1.3]
        \draw[gray] (-1.5,0) -- (1.5,0);
        \draw[gray] (0,1.5) -- (0,-1.5);
        \end{tikzpicture} 
        \end{left}
        
    $\lim_{x\to\infty} f(x)$ =
    \newline
    $\lim_{x\to-\infty} f(x)$ =
    \newline
    Vertical Asymptote(s): \line(1,0){40} \\
    \vspace{\stretch{0.5}}
   
    \question[1] Sketch the graph of the rational function $\frac{3x - 3}{x - 5}$ and find its end behavior and vertical asymptote(s).
    
        \begin{left} 
        \begin{tikzpicture}[scale=1.3]
        \draw[gray] (-1.5,0) -- (1.5,0);
        \draw[gray] (0,1.5) -- (0,-1.5);
        \end{tikzpicture} 
        \end{left}
        
    $\lim_{x\to\infty} f(x)$ =
    \newline
    $\lim_{x\to-\infty} f(x)$ =
    \newline
    Vertical Asymptote(s): \line(1,0){40} \\
    \vspace{\stretch{0.5}}
    
    \newpage
    % Unit 4
    \question[1] Sketch the graph of the function $f(x) = \frac{x^2 - 5}{x^2 - 4}$ and state the vertical and/or horizontal asymptote(s), slant asymptote(s), domain, range and end behavior.
    
        \begin{left} 
        \begin{tikzpicture}[scale=1.3]
        \draw[gray] (-1.5,0) -- (1.5,0);
        \draw[gray] (0,1.5) -- (0,-1.5);
        \end{tikzpicture} 
        \end{left}
    
    Vertical Asymptote(s): \line(1,0){40} \\
    \newline
    Horizontal Asymptote(s): \line(1,0){40} \\
    \newline
    Slant Asymptote(s): \line(1,0){40} \\
    \newline
    Domain: \line(1,0){40} \\
    \newline
    Range: \line(1,0){40} \\
    \newline
    $\lim_{x\to\infty} f(x)$: \line(1,0){40} \\
    \newline
    $\lim_{x\to-\infty} f(x)$: \line(1,0){40} \\
    \vspace{\stretch{0.5}}
    
    \question[1] Sketch the graph of the function $f(x) = \frac{-4}{x^2-8x+3}$ and state the vertical and/or horizontal asymptote(s), slant asymptote(s), domain, range and end behavior.
    
        \begin{left} 
        \begin{tikzpicture}[scale=1.3]
        \draw[gray] (-1.5,0) -- (1.5,0);
        \draw[gray] (0,1.5) -- (0,-1.5);
        \end{tikzpicture} 
        \end{left}
    
    Vertical Asymptote(s): \line(1,0){40} \\
    \newline
    Horizontal Asymptote(s): \line(1,0){40} \\
    \newline
    Slant Asymptote(s): \line(1,0){40} \\
    \newline
    Domain: \line(1,0){40} \\
    \newline
    Range: \line(1,0){40} \\
    \newline
    $\lim_{x\to\infty} f(x)$: \line(1,0){40} \\
    \newline
    $\lim_{x\to-\infty} f(x)$: \line(1,0){40} \\
    \vspace{\stretch{0.5}}
    
    \newpage
    \question[1] Sketch the graph of the function $f(x) = \frac{x+4}{x+3}$ and state the vertical and/or horizontal asymptote(s), slant asymptote(s), domain, range and end behavior.
    
        \begin{left} 
        \begin{tikzpicture}[scale=1.3]
        \draw[gray] (-1.5,0) -- (1.5,0);
        \draw[gray] (0,1.5) -- (0,-1.5);
        \end{tikzpicture} 
        \end{left}
    
    Vertical Asymptote(s): \line(1,0){40} \\
    \newline
    Horizontal Asymptote(s): \line(1,0){40} \\
    \newline
    Slant Asymptote(s): \line(1,0){40} \\
    \newline
    Domain: \line(1,0){40} \\
    \newline
    Range: \line(1,0){40} \\
    \newline
    $\lim_{x\to\infty} f(x)$: \line(1,0){40} \\
    \newline
    $\lim_{x\to-\infty} f(x)$: \line(1,0){40} \\
    \vspace{\stretch{0.5}}
    
    \question[1] Sketch the graph of the function $f(x) = \frac{5}{x^2 - 16}$ and state the vertical and/or horizontal asymptote(s), slant asymptote(s), domain, range and end behavior.
    
        \begin{left} 
        \begin{tikzpicture}[scale=1.3]
        \draw[gray] (-1.5,0) -- (1.5,0);
        \draw[gray] (0,1.5) -- (0,-1.5);
        \end{tikzpicture} 
        \end{left}
    
    Vertical Asymptote(s): \line(1,0){40} \\
    \newline
    Horizontal Asymptote(s): \line(1,0){40} \\
    \newline
    Slant Asymptote(s): \line(1,0){40} \\
    \newline
    Domain: \line(1,0){40} \\
    \newline
    Range: \line(1,0){40} \\
    \newline
    $\lim_{x\to\infty} f(x)$: \line(1,0){40} \\
    \newline
    $\lim_{x\to-\infty} f(x)$: \line(1,0){40} \\
    \vspace{\stretch{0.5}}
    
    \newpage
    % Unit 5
    \question[1] Given the function $f(x) = e^x + 5$, draw a sketch, state the vertical asymptote (as $x$ =) or horizontal asymptote (as $y$ =), domain, range and end behavior.
    
        \begin{left} 
        \begin{tikzpicture}[scale=1.3]
        \draw[gray] (-1.5,0) -- (1.5,0);
        \draw[gray] (0,1.5) -- (0,-1.5);
        \end{tikzpicture} 
        \end{left}
    
    Vertical Asymptote/Horizontal Asymptote: \line(1,0){40} \\
    \newline
    Domain: \line(1,0){40} \\
    \newline
    Range: \line(1,0){40} \\
    \newline
    $\lim_{x\to\infty} f(x)$: \line(1,0){40} \\
    \newline
    $\lim_{x\to-\infty} f(x)$: \line(1,0){40} \\
    \vspace{\stretch{0.5}}
    
    \question[1] Given the function $f(x) = 4^{x+5}$, draw a sketch, state the vertical asymptote (as $x$ =) or horizontal asymptote (as $y$ =), domain, range and end behavior.
    
        \begin{left} 
        \begin{tikzpicture}[scale=1.3]
        \draw[gray] (-1.5,0) -- (1.5,0);
        \draw[gray] (0,1.5) -- (0,-1.5);
        \end{tikzpicture} 
        \end{left}
    
    Vertical Asymptote/Horizontal Asymptote: \line(1,0){40} \\
    \newline
    Domain: \line(1,0){40} \\
    \newline
    Range: \line(1,0){40} \\
    \newline
    $\lim_{x\to\infty} f(x)$: \line(1,0){40} \\
    \newline
    $\lim_{x\to-\infty} f(x)$: \line(1,0){40} \\
    \vspace{\stretch{0.5}}
    
    \newpage
    \question[1] Given the function $f(x) = (\frac{3}{4})^x + 4$, draw a sketch, state the vertical asymptote (as $x$ =) or horizontal asymptote (as $y$ =), domain, range and end behavior.
    
        \begin{left} 
        \begin{tikzpicture}[scale=1.3]
        \draw[gray] (-1.5,0) -- (1.5,0);
        \draw[gray] (0,1.5) -- (0,-1.5);
        \end{tikzpicture} 
        \end{left}
    
    Vertical Asymptote/Horizontal Asymptote: \line(1,0){40} \\
    \newline
    Domain: \line(1,0){40} \\
    \newline
    Range: \line(1,0){40} \\
    \newline
    $\lim_{x\to\infty} f(x)$: \line(1,0){40} \\
    \newline
    $\lim_{x\to-\infty} f(x)$: \line(1,0){40} \\
    \vspace{\stretch{0.5}}
    
    \question[1] Given the function $f(x) = \log (x-2) + 1$, sketch a graph and state the vertical and horizontal asymptote(s), slant asymptote(s), domain, range and end behavior.
    
        \begin{left} 
        \begin{tikzpicture}[scale=1.3]
        \draw[gray] (-1.5,0) -- (1.5,0);
        \draw[gray] (0,1.5) -- (0,-1.5);
        \end{tikzpicture} 
        \end{left}
    
    Vertical Asymptote(s): \line(1,0){40} \\
    \newline
    Horizontal Asymptote(s): \line(1,0){40} \\
    \newline
    Slant Asymptote(s): \line(1,0){40} \\
    \newline
    Domain: \line(1,0){40} \\
    \newline
    Range: \line(1,0){40} \\
    \newline
    $\lim_{x\to\infty} f(x)$: \line(1,0){40} \\
    \newline
    $\lim_{x\to-\infty} f(x)$: \line(1,0){40} \\
    \vspace{\stretch{0.5}}
    
    \newpage
    \question[1] Given the function $f(x) = \ln(x) + 3$, sketch a graph and state the vertical and horizontal asymptote(s), slant asymptote(s), domain, range and end behavior.
    
        \begin{left} 
        \begin{tikzpicture}[scale=1.3]
        \draw[gray] (-1.5,0) -- (1.5,0);
        \draw[gray] (0,1.5) -- (0,-1.5);
        \end{tikzpicture} 
        \end{left}
    
    Vertical Asymptote(s): \line(1,0){40} \\
    \newline
    Horizontal Asymptote(s): \line(1,0){40} \\
    \newline
    Slant Asymptote(s): \line(1,0){40} \\
    \newline
    Domain: \line(1,0){40} \\
    \newline
    Range: \line(1,0){40} \\
    \newline
    $\lim_{x\to\infty} f(x)$: \line(1,0){40} \\
    \newline
    $\lim_{x\to-\infty} f(x)$: \line(1,0){40} \\
    \vspace{\stretch{0.5}}
    
    \question[1] Given the function $f(x) = 3\log(x-6)$, sketch a graph and state the vertical and horizontal asymptote(s), slant asymptote(s), domain, range and end behavior.
    
        \begin{left} 
        \begin{tikzpicture}[scale=1.3]
        \draw[gray] (-1.5,0) -- (1.5,0);
        \draw[gray] (0,1.5) -- (0,-1.5);
        \end{tikzpicture} 
        \end{left}
    
    Vertical Asymptote(s): \line(1,0){40} \\
    \newline
    Horizontal Asymptote(s): \line(1,0){40} \\
    \newline
    Slant Asymptote(s): \line(1,0){40} \\
    \newline
    Domain: \line(1,0){40} \\
    \newline
    Range: \line(1,0){40} \\
    \newline
    $\lim_{x\to\infty} f(x)$: \line(1,0){40} \\
    \newline
    $\lim_{x\to-\infty} f(x)$: \line(1,0){40} \\
    \vspace{\stretch{0.5}}
    
    \newpage
    % Unit 6
    \question[1] Graph $r = 4\sin(\theta) + 3$
    
        \begin{left} 
        \begin{tikzpicture}[scale=1.3]
        \draw[gray] (-1.5,0) -- (1.5,0);
        \draw[gray] (0,1.5) -- (0,-1.5);
        \end{tikzpicture} 
        \end{left}
    
    Range: \line(1,0){40} \\
    \newline
    Domain: \line(1,0){40} \\
    \newline
    Period: \line(1,0){40} \\
    \vspace{\stretch{0.5}}
    
    \question[1] Graph $r = 5\csc(\theta)- 4$
    
        \begin{left} 
        \begin{tikzpicture}[scale=1.3]
        \draw[gray] (-1.5,0) -- (1.5,0);
        \draw[gray] (0,1.5) -- (0,-1.5);
        \end{tikzpicture} 
        \end{left}
    
    Range: \line(1,0){40} \\
    \newline
    Period: \line(1,0){40} \\
    \vspace{\stretch{0.5}}
    
    \question[1] Graph $r = \sec(\theta + \frac{\pi}{4})+3$
    
        \begin{left} 
        \begin{tikzpicture}[scale=1.3]
        \draw[gray] (-1.5,0) -- (1.5,0);
        \draw[gray] (0,1.5) -- (0,-1.5);
        \end{tikzpicture} 
        \end{left}
    
    Range: \line(1,0){40} \\
    \newline
    Period: \line(1,0){40} \\
    \vspace{\stretch{0.5}}
    
    \newpage
    \question[1] Graph $r = 5(\sin(\theta))^2$
    
        \begin{left} 
        \begin{tikzpicture}[scale=1.3]
        \draw[gray] (-1.5,0) -- (1.5,0);
        \draw[gray] (0,1.5) -- (0,-1.5);
        \end{tikzpicture} 
        \end{left}
    
    Amplitude: \line(1,0){40} \\
    \newline
    Domain: \line(1,0){40} \\
    \newline
    Range: \line(1,0){40} \\
    \newline
    Period: \line(1,0){40} \\
    \vspace{\stretch{0.5}}
    
    % Unit 7: N/A
    
    % Unit 8
    \question[1] Sketch the function $r = 4\cos 2\theta$ and analyze the graph of the polar curve.
    
        \begin{left} 
        \begin{tikzpicture}[scale=1.3]
        \draw[gray] (-1.5,0) -- (1.5,0);
        \draw[gray] (0,1.5) -- (0,-1.5);
        \end{tikzpicture} 
        \end{left}
    
    Continuous: \line(1,0){40} \\
    \newline
    Symmetric: \line(1,0){40} \\
    \newline
    Domain: \line(1,0){40} \\
    \newline
    Range: \line(1,0){40} \\
    \vspace{\stretch{0.5}}
    
    \newpage
    \question[1] Sketch the function $r = 5 + \sin\theta$
    
        \begin{left} 
        \begin{tikzpicture}[scale=1.3]
        \draw[gray] (-1.5,0) -- (1.5,0);
        \draw[gray] (0,1.5) -- (0,-1.5);
        \end{tikzpicture} 
        \end{left}
    
    Continuous: \line(1,0){40} \\
    \newline
    Symmetric: \line(1,0){40} \\
    \newline
    Domain: \line(1,0){40} \\
    \newline
    Range: \line(1,0){40} \\
    \vspace{\stretch{0.5}}
    
    \question[1] Sketch the function $r = 7\cos3\theta$ and analyze the graph of the polar curve.
    
        \begin{left} 
        \begin{tikzpicture}[scale=1.3]
        \draw[gray] (-1.5,0) -- (1.5,0);
        \draw[gray] (0,1.5) -- (0,-1.5);
        \end{tikzpicture} 
        \end{left}
    
    Continuous: \line(1,0){40} \\
    \newline
    Symmetric: \line(1,0){40} \\
    \newline
    Domain: \line(1,0){40} \\
    \newline
    Range: \line(1,0){40} \\
    \vspace{\stretch{0.5}}
    
    \newpage
    \question[1] Sketch the function $r = 4\cos2\theta$ and analyze the graph of the polar curve.
    
        \begin{left} 
        \begin{tikzpicture}[scale=1.3]
        \draw[gray] (-1.5,0) -- (1.5,0);
        \draw[gray] (0,1.5) -- (0,-1.5);
        \end{tikzpicture} 
        \end{left}
    
    Continuous: \line(1,0){40} \\
    \newline
    Symmetric: \line(1,0){40} \\
    \newline
    Domain: \line(1,0){40} \\
    \newline
    Range: \line(1,0){40} \\
    \vspace{\stretch{0.5}}
    
    \question[1] Sketch the function $r = 8\cos4\theta$ and analyze the graph of the polar curve.
    
        \begin{left} 
        \begin{tikzpicture}[scale=1.3]
        \draw[gray] (-1.5,0) -- (1.5,0);
        \draw[gray] (0,1.5) -- (0,-1.5);
        \end{tikzpicture} 
        \end{left}
    
    Continuous: \line(1,0){40} \\
    \newline
    Symmetric: \line(1,0){40} \\
    \newline
    Domain: \line(1,0){40} \\
    \newline
    Range: \line(1,0){40} \\
    \vspace{\stretch{0.5}}
    
    \newpage
    \question[1] Sketch the function $r = 5\cos2\theta$ and analyze the graph of the polar curve.
    
        \begin{left} 
        \begin{tikzpicture}[scale=1.3]
        \draw[gray] (-1.5,0) -- (1.5,0);
        \draw[gray] (0,1.5) -- (0,-1.5);
        \end{tikzpicture} 
        \end{left}
    
    Continuous: \line(1,0){40} \\
    \newline
    Symmetric: \line(1,0){40} \\
    \newline
    Domain: \line(1,0){40} \\
    \newline
    Range: \line(1,0){40} \\
    \vspace{\stretch{0.5}}
    \end{questions}
\end{document}